
% Default to the notebook output style

    


% Inherit from the specified cell style.




    
\documentclass[11pt]{article}

    
    
    \usepackage[T1]{fontenc}
    % Nicer default font (+ math font) than Computer Modern for most use cases
    \usepackage{mathpazo}

    % Basic figure setup, for now with no caption control since it's done
    % automatically by Pandoc (which extracts ![](path) syntax from Markdown).
    \usepackage{graphicx}
    % We will generate all images so they have a width \maxwidth. This means
    % that they will get their normal width if they fit onto the page, but
    % are scaled down if they would overflow the margins.
    \makeatletter
    \def\maxwidth{\ifdim\Gin@nat@width>\linewidth\linewidth
    \else\Gin@nat@width\fi}
    \makeatother
    \let\Oldincludegraphics\includegraphics
    % Set max figure width to be 80% of text width, for now hardcoded.
    \renewcommand{\includegraphics}[1]{\Oldincludegraphics[width=.8\maxwidth]{#1}}
    % Ensure that by default, figures have no caption (until we provide a
    % proper Figure object with a Caption API and a way to capture that
    % in the conversion process - todo).
    \usepackage{caption}
    \DeclareCaptionLabelFormat{nolabel}{}
    \captionsetup{labelformat=nolabel}

    \usepackage{adjustbox} % Used to constrain images to a maximum size 
    \usepackage{xcolor} % Allow colors to be defined
    \usepackage{enumerate} % Needed for markdown enumerations to work
    \usepackage{geometry} % Used to adjust the document margins
    \usepackage{amsmath} % Equations
    \usepackage{amssymb} % Equations
    \usepackage{textcomp} % defines textquotesingle
    % Hack from http://tex.stackexchange.com/a/47451/13684:
    \AtBeginDocument{%
        \def\PYZsq{\textquotesingle}% Upright quotes in Pygmentized code
    }
    \usepackage{upquote} % Upright quotes for verbatim code
    \usepackage{eurosym} % defines \euro
    \usepackage[mathletters]{ucs} % Extended unicode (utf-8) support
    \usepackage[utf8x]{inputenc} % Allow utf-8 characters in the tex document
    \usepackage{fancyvrb} % verbatim replacement that allows latex
    \usepackage{grffile} % extends the file name processing of package graphics 
                         % to support a larger range 
    % The hyperref package gives us a pdf with properly built
    % internal navigation ('pdf bookmarks' for the table of contents,
    % internal cross-reference links, web links for URLs, etc.)
    \usepackage{hyperref}
    \usepackage{longtable} % longtable support required by pandoc >1.10
    \usepackage{booktabs}  % table support for pandoc > 1.12.2
    \usepackage[inline]{enumitem} % IRkernel/repr support (it uses the enumerate* environment)
    \usepackage[normalem]{ulem} % ulem is needed to support strikethroughs (\sout)
                                % normalem makes italics be italics, not underlines
    

    
    
    % Colors for the hyperref package
    \definecolor{urlcolor}{rgb}{0,.145,.698}
    \definecolor{linkcolor}{rgb}{.71,0.21,0.01}
    \definecolor{citecolor}{rgb}{.12,.54,.11}

    % ANSI colors
    \definecolor{ansi-black}{HTML}{3E424D}
    \definecolor{ansi-black-intense}{HTML}{282C36}
    \definecolor{ansi-red}{HTML}{E75C58}
    \definecolor{ansi-red-intense}{HTML}{B22B31}
    \definecolor{ansi-green}{HTML}{00A250}
    \definecolor{ansi-green-intense}{HTML}{007427}
    \definecolor{ansi-yellow}{HTML}{DDB62B}
    \definecolor{ansi-yellow-intense}{HTML}{B27D12}
    \definecolor{ansi-blue}{HTML}{208FFB}
    \definecolor{ansi-blue-intense}{HTML}{0065CA}
    \definecolor{ansi-magenta}{HTML}{D160C4}
    \definecolor{ansi-magenta-intense}{HTML}{A03196}
    \definecolor{ansi-cyan}{HTML}{60C6C8}
    \definecolor{ansi-cyan-intense}{HTML}{258F8F}
    \definecolor{ansi-white}{HTML}{C5C1B4}
    \definecolor{ansi-white-intense}{HTML}{A1A6B2}

    % commands and environments needed by pandoc snippets
    % extracted from the output of `pandoc -s`
    \providecommand{\tightlist}{%
      \setlength{\itemsep}{0pt}\setlength{\parskip}{0pt}}
    \DefineVerbatimEnvironment{Highlighting}{Verbatim}{commandchars=\\\{\}}
    % Add ',fontsize=\small' for more characters per line
    \newenvironment{Shaded}{}{}
    \newcommand{\KeywordTok}[1]{\textcolor[rgb]{0.00,0.44,0.13}{\textbf{{#1}}}}
    \newcommand{\DataTypeTok}[1]{\textcolor[rgb]{0.56,0.13,0.00}{{#1}}}
    \newcommand{\DecValTok}[1]{\textcolor[rgb]{0.25,0.63,0.44}{{#1}}}
    \newcommand{\BaseNTok}[1]{\textcolor[rgb]{0.25,0.63,0.44}{{#1}}}
    \newcommand{\FloatTok}[1]{\textcolor[rgb]{0.25,0.63,0.44}{{#1}}}
    \newcommand{\CharTok}[1]{\textcolor[rgb]{0.25,0.44,0.63}{{#1}}}
    \newcommand{\StringTok}[1]{\textcolor[rgb]{0.25,0.44,0.63}{{#1}}}
    \newcommand{\CommentTok}[1]{\textcolor[rgb]{0.38,0.63,0.69}{\textit{{#1}}}}
    \newcommand{\OtherTok}[1]{\textcolor[rgb]{0.00,0.44,0.13}{{#1}}}
    \newcommand{\AlertTok}[1]{\textcolor[rgb]{1.00,0.00,0.00}{\textbf{{#1}}}}
    \newcommand{\FunctionTok}[1]{\textcolor[rgb]{0.02,0.16,0.49}{{#1}}}
    \newcommand{\RegionMarkerTok}[1]{{#1}}
    \newcommand{\ErrorTok}[1]{\textcolor[rgb]{1.00,0.00,0.00}{\textbf{{#1}}}}
    \newcommand{\NormalTok}[1]{{#1}}
    
    % Additional commands for more recent versions of Pandoc
    \newcommand{\ConstantTok}[1]{\textcolor[rgb]{0.53,0.00,0.00}{{#1}}}
    \newcommand{\SpecialCharTok}[1]{\textcolor[rgb]{0.25,0.44,0.63}{{#1}}}
    \newcommand{\VerbatimStringTok}[1]{\textcolor[rgb]{0.25,0.44,0.63}{{#1}}}
    \newcommand{\SpecialStringTok}[1]{\textcolor[rgb]{0.73,0.40,0.53}{{#1}}}
    \newcommand{\ImportTok}[1]{{#1}}
    \newcommand{\DocumentationTok}[1]{\textcolor[rgb]{0.73,0.13,0.13}{\textit{{#1}}}}
    \newcommand{\AnnotationTok}[1]{\textcolor[rgb]{0.38,0.63,0.69}{\textbf{\textit{{#1}}}}}
    \newcommand{\CommentVarTok}[1]{\textcolor[rgb]{0.38,0.63,0.69}{\textbf{\textit{{#1}}}}}
    \newcommand{\VariableTok}[1]{\textcolor[rgb]{0.10,0.09,0.49}{{#1}}}
    \newcommand{\ControlFlowTok}[1]{\textcolor[rgb]{0.00,0.44,0.13}{\textbf{{#1}}}}
    \newcommand{\OperatorTok}[1]{\textcolor[rgb]{0.40,0.40,0.40}{{#1}}}
    \newcommand{\BuiltInTok}[1]{{#1}}
    \newcommand{\ExtensionTok}[1]{{#1}}
    \newcommand{\PreprocessorTok}[1]{\textcolor[rgb]{0.74,0.48,0.00}{{#1}}}
    \newcommand{\AttributeTok}[1]{\textcolor[rgb]{0.49,0.56,0.16}{{#1}}}
    \newcommand{\InformationTok}[1]{\textcolor[rgb]{0.38,0.63,0.69}{\textbf{\textit{{#1}}}}}
    \newcommand{\WarningTok}[1]{\textcolor[rgb]{0.38,0.63,0.69}{\textbf{\textit{{#1}}}}}
    
    
    % Define a nice break command that doesn't care if a line doesn't already
    % exist.
    \def\br{\hspace*{\fill} \\* }
    % Math Jax compatability definitions
    \def\gt{>}
    \def\lt{<}
    % Document parameters
    \title{B\_TernaryBody\_ParameterScan\_DissociationConstant}
    
    
    

    % Pygments definitions
    
\makeatletter
\def\PY@reset{\let\PY@it=\relax \let\PY@bf=\relax%
    \let\PY@ul=\relax \let\PY@tc=\relax%
    \let\PY@bc=\relax \let\PY@ff=\relax}
\def\PY@tok#1{\csname PY@tok@#1\endcsname}
\def\PY@toks#1+{\ifx\relax#1\empty\else%
    \PY@tok{#1}\expandafter\PY@toks\fi}
\def\PY@do#1{\PY@bc{\PY@tc{\PY@ul{%
    \PY@it{\PY@bf{\PY@ff{#1}}}}}}}
\def\PY#1#2{\PY@reset\PY@toks#1+\relax+\PY@do{#2}}

\expandafter\def\csname PY@tok@il\endcsname{\def\PY@tc##1{\textcolor[rgb]{0.40,0.40,0.40}{##1}}}
\expandafter\def\csname PY@tok@sd\endcsname{\let\PY@it=\textit\def\PY@tc##1{\textcolor[rgb]{0.73,0.13,0.13}{##1}}}
\expandafter\def\csname PY@tok@mf\endcsname{\def\PY@tc##1{\textcolor[rgb]{0.40,0.40,0.40}{##1}}}
\expandafter\def\csname PY@tok@no\endcsname{\def\PY@tc##1{\textcolor[rgb]{0.53,0.00,0.00}{##1}}}
\expandafter\def\csname PY@tok@vg\endcsname{\def\PY@tc##1{\textcolor[rgb]{0.10,0.09,0.49}{##1}}}
\expandafter\def\csname PY@tok@kd\endcsname{\let\PY@bf=\textbf\def\PY@tc##1{\textcolor[rgb]{0.00,0.50,0.00}{##1}}}
\expandafter\def\csname PY@tok@nf\endcsname{\def\PY@tc##1{\textcolor[rgb]{0.00,0.00,1.00}{##1}}}
\expandafter\def\csname PY@tok@mh\endcsname{\def\PY@tc##1{\textcolor[rgb]{0.40,0.40,0.40}{##1}}}
\expandafter\def\csname PY@tok@si\endcsname{\let\PY@bf=\textbf\def\PY@tc##1{\textcolor[rgb]{0.73,0.40,0.53}{##1}}}
\expandafter\def\csname PY@tok@s\endcsname{\def\PY@tc##1{\textcolor[rgb]{0.73,0.13,0.13}{##1}}}
\expandafter\def\csname PY@tok@nn\endcsname{\let\PY@bf=\textbf\def\PY@tc##1{\textcolor[rgb]{0.00,0.00,1.00}{##1}}}
\expandafter\def\csname PY@tok@nl\endcsname{\def\PY@tc##1{\textcolor[rgb]{0.63,0.63,0.00}{##1}}}
\expandafter\def\csname PY@tok@na\endcsname{\def\PY@tc##1{\textcolor[rgb]{0.49,0.56,0.16}{##1}}}
\expandafter\def\csname PY@tok@sx\endcsname{\def\PY@tc##1{\textcolor[rgb]{0.00,0.50,0.00}{##1}}}
\expandafter\def\csname PY@tok@cs\endcsname{\let\PY@it=\textit\def\PY@tc##1{\textcolor[rgb]{0.25,0.50,0.50}{##1}}}
\expandafter\def\csname PY@tok@cpf\endcsname{\let\PY@it=\textit\def\PY@tc##1{\textcolor[rgb]{0.25,0.50,0.50}{##1}}}
\expandafter\def\csname PY@tok@ow\endcsname{\let\PY@bf=\textbf\def\PY@tc##1{\textcolor[rgb]{0.67,0.13,1.00}{##1}}}
\expandafter\def\csname PY@tok@kt\endcsname{\def\PY@tc##1{\textcolor[rgb]{0.69,0.00,0.25}{##1}}}
\expandafter\def\csname PY@tok@s1\endcsname{\def\PY@tc##1{\textcolor[rgb]{0.73,0.13,0.13}{##1}}}
\expandafter\def\csname PY@tok@fm\endcsname{\def\PY@tc##1{\textcolor[rgb]{0.00,0.00,1.00}{##1}}}
\expandafter\def\csname PY@tok@kr\endcsname{\let\PY@bf=\textbf\def\PY@tc##1{\textcolor[rgb]{0.00,0.50,0.00}{##1}}}
\expandafter\def\csname PY@tok@nv\endcsname{\def\PY@tc##1{\textcolor[rgb]{0.10,0.09,0.49}{##1}}}
\expandafter\def\csname PY@tok@sh\endcsname{\def\PY@tc##1{\textcolor[rgb]{0.73,0.13,0.13}{##1}}}
\expandafter\def\csname PY@tok@m\endcsname{\def\PY@tc##1{\textcolor[rgb]{0.40,0.40,0.40}{##1}}}
\expandafter\def\csname PY@tok@ge\endcsname{\let\PY@it=\textit}
\expandafter\def\csname PY@tok@sb\endcsname{\def\PY@tc##1{\textcolor[rgb]{0.73,0.13,0.13}{##1}}}
\expandafter\def\csname PY@tok@vm\endcsname{\def\PY@tc##1{\textcolor[rgb]{0.10,0.09,0.49}{##1}}}
\expandafter\def\csname PY@tok@mo\endcsname{\def\PY@tc##1{\textcolor[rgb]{0.40,0.40,0.40}{##1}}}
\expandafter\def\csname PY@tok@gr\endcsname{\def\PY@tc##1{\textcolor[rgb]{1.00,0.00,0.00}{##1}}}
\expandafter\def\csname PY@tok@gp\endcsname{\let\PY@bf=\textbf\def\PY@tc##1{\textcolor[rgb]{0.00,0.00,0.50}{##1}}}
\expandafter\def\csname PY@tok@sc\endcsname{\def\PY@tc##1{\textcolor[rgb]{0.73,0.13,0.13}{##1}}}
\expandafter\def\csname PY@tok@nc\endcsname{\let\PY@bf=\textbf\def\PY@tc##1{\textcolor[rgb]{0.00,0.00,1.00}{##1}}}
\expandafter\def\csname PY@tok@go\endcsname{\def\PY@tc##1{\textcolor[rgb]{0.53,0.53,0.53}{##1}}}
\expandafter\def\csname PY@tok@w\endcsname{\def\PY@tc##1{\textcolor[rgb]{0.73,0.73,0.73}{##1}}}
\expandafter\def\csname PY@tok@kc\endcsname{\let\PY@bf=\textbf\def\PY@tc##1{\textcolor[rgb]{0.00,0.50,0.00}{##1}}}
\expandafter\def\csname PY@tok@ss\endcsname{\def\PY@tc##1{\textcolor[rgb]{0.10,0.09,0.49}{##1}}}
\expandafter\def\csname PY@tok@gt\endcsname{\def\PY@tc##1{\textcolor[rgb]{0.00,0.27,0.87}{##1}}}
\expandafter\def\csname PY@tok@gi\endcsname{\def\PY@tc##1{\textcolor[rgb]{0.00,0.63,0.00}{##1}}}
\expandafter\def\csname PY@tok@kp\endcsname{\def\PY@tc##1{\textcolor[rgb]{0.00,0.50,0.00}{##1}}}
\expandafter\def\csname PY@tok@se\endcsname{\let\PY@bf=\textbf\def\PY@tc##1{\textcolor[rgb]{0.73,0.40,0.13}{##1}}}
\expandafter\def\csname PY@tok@nb\endcsname{\def\PY@tc##1{\textcolor[rgb]{0.00,0.50,0.00}{##1}}}
\expandafter\def\csname PY@tok@nt\endcsname{\let\PY@bf=\textbf\def\PY@tc##1{\textcolor[rgb]{0.00,0.50,0.00}{##1}}}
\expandafter\def\csname PY@tok@cp\endcsname{\def\PY@tc##1{\textcolor[rgb]{0.74,0.48,0.00}{##1}}}
\expandafter\def\csname PY@tok@vc\endcsname{\def\PY@tc##1{\textcolor[rgb]{0.10,0.09,0.49}{##1}}}
\expandafter\def\csname PY@tok@o\endcsname{\def\PY@tc##1{\textcolor[rgb]{0.40,0.40,0.40}{##1}}}
\expandafter\def\csname PY@tok@k\endcsname{\let\PY@bf=\textbf\def\PY@tc##1{\textcolor[rgb]{0.00,0.50,0.00}{##1}}}
\expandafter\def\csname PY@tok@mi\endcsname{\def\PY@tc##1{\textcolor[rgb]{0.40,0.40,0.40}{##1}}}
\expandafter\def\csname PY@tok@gu\endcsname{\let\PY@bf=\textbf\def\PY@tc##1{\textcolor[rgb]{0.50,0.00,0.50}{##1}}}
\expandafter\def\csname PY@tok@ch\endcsname{\let\PY@it=\textit\def\PY@tc##1{\textcolor[rgb]{0.25,0.50,0.50}{##1}}}
\expandafter\def\csname PY@tok@sa\endcsname{\def\PY@tc##1{\textcolor[rgb]{0.73,0.13,0.13}{##1}}}
\expandafter\def\csname PY@tok@dl\endcsname{\def\PY@tc##1{\textcolor[rgb]{0.73,0.13,0.13}{##1}}}
\expandafter\def\csname PY@tok@ne\endcsname{\let\PY@bf=\textbf\def\PY@tc##1{\textcolor[rgb]{0.82,0.25,0.23}{##1}}}
\expandafter\def\csname PY@tok@cm\endcsname{\let\PY@it=\textit\def\PY@tc##1{\textcolor[rgb]{0.25,0.50,0.50}{##1}}}
\expandafter\def\csname PY@tok@kn\endcsname{\let\PY@bf=\textbf\def\PY@tc##1{\textcolor[rgb]{0.00,0.50,0.00}{##1}}}
\expandafter\def\csname PY@tok@err\endcsname{\def\PY@bc##1{\setlength{\fboxsep}{0pt}\fcolorbox[rgb]{1.00,0.00,0.00}{1,1,1}{\strut ##1}}}
\expandafter\def\csname PY@tok@vi\endcsname{\def\PY@tc##1{\textcolor[rgb]{0.10,0.09,0.49}{##1}}}
\expandafter\def\csname PY@tok@sr\endcsname{\def\PY@tc##1{\textcolor[rgb]{0.73,0.40,0.53}{##1}}}
\expandafter\def\csname PY@tok@c1\endcsname{\let\PY@it=\textit\def\PY@tc##1{\textcolor[rgb]{0.25,0.50,0.50}{##1}}}
\expandafter\def\csname PY@tok@s2\endcsname{\def\PY@tc##1{\textcolor[rgb]{0.73,0.13,0.13}{##1}}}
\expandafter\def\csname PY@tok@gh\endcsname{\let\PY@bf=\textbf\def\PY@tc##1{\textcolor[rgb]{0.00,0.00,0.50}{##1}}}
\expandafter\def\csname PY@tok@c\endcsname{\let\PY@it=\textit\def\PY@tc##1{\textcolor[rgb]{0.25,0.50,0.50}{##1}}}
\expandafter\def\csname PY@tok@mb\endcsname{\def\PY@tc##1{\textcolor[rgb]{0.40,0.40,0.40}{##1}}}
\expandafter\def\csname PY@tok@gs\endcsname{\let\PY@bf=\textbf}
\expandafter\def\csname PY@tok@ni\endcsname{\let\PY@bf=\textbf\def\PY@tc##1{\textcolor[rgb]{0.60,0.60,0.60}{##1}}}
\expandafter\def\csname PY@tok@gd\endcsname{\def\PY@tc##1{\textcolor[rgb]{0.63,0.00,0.00}{##1}}}
\expandafter\def\csname PY@tok@bp\endcsname{\def\PY@tc##1{\textcolor[rgb]{0.00,0.50,0.00}{##1}}}
\expandafter\def\csname PY@tok@nd\endcsname{\def\PY@tc##1{\textcolor[rgb]{0.67,0.13,1.00}{##1}}}

\def\PYZbs{\char`\\}
\def\PYZus{\char`\_}
\def\PYZob{\char`\{}
\def\PYZcb{\char`\}}
\def\PYZca{\char`\^}
\def\PYZam{\char`\&}
\def\PYZlt{\char`\<}
\def\PYZgt{\char`\>}
\def\PYZsh{\char`\#}
\def\PYZpc{\char`\%}
\def\PYZdl{\char`\$}
\def\PYZhy{\char`\-}
\def\PYZsq{\char`\'}
\def\PYZdq{\char`\"}
\def\PYZti{\char`\~}
% for compatibility with earlier versions
\def\PYZat{@}
\def\PYZlb{[}
\def\PYZrb{]}
\makeatother


    % Exact colors from NB
    \definecolor{incolor}{rgb}{0.0, 0.0, 0.5}
    \definecolor{outcolor}{rgb}{0.545, 0.0, 0.0}



    
    % Prevent overflowing lines due to hard-to-break entities
    \sloppy 
    % Setup hyperref package
    \hypersetup{
      breaklinks=true,  % so long urls are correctly broken across lines
      colorlinks=true,
      urlcolor=urlcolor,
      linkcolor=linkcolor,
      citecolor=citecolor,
      }
    % Slightly bigger margins than the latex defaults
    
    \geometry{verbose,tmargin=1in,bmargin=1in,lmargin=1in,rmargin=1in}
    
    

    \begin{document}
    
    
    \maketitle
    
    

    
    \hypertarget{scanning-parameters-in-t-cell-ternary-body-formation}{%
\section{Scanning Parameters in T Cell Ternary Body
Formation}\label{scanning-parameters-in-t-cell-ternary-body-formation}}

\hypertarget{analyzing-antibody-dissociation-from-covalently-bonded-snap}{%
\subsection{Analyzing antibody dissociation from covalently bonded
SNAP}\label{analyzing-antibody-dissociation-from-covalently-bonded-snap}}

This is a script for analyzing the importance of the antibody
dissociation rate from a CAR T Cell or a SNAP T Cell. Where as the setup
script assumed no dissociation between T Cells and the antibody, we will
be scanning the parameter space for optimum dissociation rates.

\hypertarget{importing-the-important-packages-to-simulate-the-model.}{%
\subsubsection{Importing the important packages to simulate the
model.}\label{importing-the-important-packages-to-simulate-the-model.}}

    \begin{Verbatim}[commandchars=\\\{\}]
{\color{incolor}In [{\color{incolor}29}]:} \PY{k+kn}{import} \PY{n+nn}{matplotlib}\PY{n+nn}{.}\PY{n+nn}{gridspec} \PY{k}{as} \PY{n+nn}{gridspec}
         \PY{k+kn}{from} \PY{n+nn}{scipy}\PY{n+nn}{.}\PY{n+nn}{integrate} \PY{k}{import} \PY{n}{odeint}
         \PY{k+kn}{import} \PY{n+nn}{matplotlib}\PY{n+nn}{.}\PY{n+nn}{pyplot} \PY{k}{as} \PY{n+nn}{plt}
         \PY{k+kn}{import} \PY{n+nn}{numpy} \PY{k}{as} \PY{n+nn}{np}
         \PY{k+kn}{import} \PY{n+nn}{math}
\end{Verbatim}


    \hypertarget{input-of-key-variables}{%
\subsubsection{Input of key variables}\label{input-of-key-variables}}

Here is where the user should define the variables they are interested
in parameter scanning and simulating. We would like to notes that these
are just some of the parameters that a user can modify and more
variables are included below. The variables included here are ones
relevant to different antibody-target pairs while those included later
are more related to the experimental setup.

    \begin{Verbatim}[commandchars=\\\{\}]
{\color{incolor}In [{\color{incolor}30}]:} \PY{c+c1}{\PYZsh{}Parameters for the parameter scan}
         \PY{n}{Dissociation\PYZus{}Rate\PYZus{}upper\PYZus{}bound} \PY{o}{=} \PY{l+m+mi}{10}\PY{o}{*}\PY{o}{*}\PY{o}{\PYZhy{}}\PY{l+m+mi}{3} \PY{c+c1}{\PYZsh{}(units = M) Dissociation constant of antibody in Molar }
         \PY{n}{Dissociation\PYZus{}Rate\PYZus{}lower\PYZus{}bound} \PY{o}{=} \PY{l+m+mi}{10}\PY{o}{*}\PY{o}{*}\PY{o}{\PYZhy{}}\PY{l+m+mi}{9} \PY{c+c1}{\PYZsh{}(units = M) Dissociation constant of antibody in Molar}
         
         \PY{c+c1}{\PYZsh{}Regular Parameters from before}
         \PY{n}{number\PYZus{}of\PYZus{}receptors\PYZus{}per\PYZus{}Tcell} \PY{o}{=} \PY{l+m+mi}{100000}  \PY{c+c1}{\PYZsh{}(units = molecules per cell) Receptors per T cell}
         \PY{n}{number\PYZus{}of\PYZus{}receptors\PYZus{}per\PYZus{}tumor} \PY{o}{=} \PY{l+m+mi}{100000}  \PY{c+c1}{\PYZsh{}(units = molecules per cell) Receptors per Tumor cell}
         \PY{n}{alpha} \PY{o}{=} \PY{l+m+mi}{10}                              \PY{c+c1}{\PYZsh{}Cooperativity rate }
                                                 \PY{c+c1}{\PYZsh{}\PYZsh{}\PYZsh{}\PYZsh{}\PYZsh{}   Cooperativity   \PYZsh{}\PYZsh{}\PYZsh{}\PYZsh{}\PYZsh{}}
                                                 \PY{c+c1}{\PYZsh{} (0 to 1)  is negative cooperativity}
                                                 \PY{c+c1}{\PYZsh{} (1)       is no cooperativitiy}
                                                 \PY{c+c1}{\PYZsh{} (1 to inifinity) is positive cooperativity}
\end{Verbatim}


    \hypertarget{calculation-of-key-variables-and-parameterization-of-the-system-of-odes}{%
\subsubsection{Calculation of key variables and parameterization of the
system of
ODEs}\label{calculation-of-key-variables-and-parameterization-of-the-system-of-odes}}

Please note that since we are interested in equilibrium concentrations,
the rates of reactions are not incredibly important; rather, the ratio
of forward rates to backward rates (association and dissociation
equilibirum constants) are what dictate equilibirum concentrations.

    \begin{Verbatim}[commandchars=\\\{\}]
{\color{incolor}In [{\color{incolor}31}]:} \PY{c+c1}{\PYZsh{}Calculating the number of steps to effectively show the parameter scan}
         \PY{n}{log\PYZus{}lower\PYZus{}bound} \PY{o}{=} \PY{n}{np}\PY{o}{.}\PY{n}{log10}\PY{p}{(}\PY{n}{Dissociation\PYZus{}Rate\PYZus{}lower\PYZus{}bound}\PY{p}{)} 
         \PY{n}{log\PYZus{}upper\PYZus{}bound} \PY{o}{=} \PY{n}{np}\PY{o}{.}\PY{n}{log10}\PY{p}{(}\PY{n}{Dissociation\PYZus{}Rate\PYZus{}upper\PYZus{}bound}\PY{p}{)} 
         \PY{n}{step\PYZus{}size} \PY{o}{=} \PY{n+nb}{int}\PY{p}{(}\PY{n+nb}{abs}\PY{p}{(}\PY{n}{log\PYZus{}lower\PYZus{}bound}\PY{o}{\PYZhy{}}\PY{n}{log\PYZus{}upper\PYZus{}bound}\PY{p}{)} \PY{o}{+} \PY{l+m+mi}{1}\PY{p}{)}
         \PY{c+c1}{\PYZsh{} step\PYZus{}size = 7                           \PYZsh{}The number of steps in the parameter scan (logarithmic)}
         
         \PY{c+c1}{\PYZsh{}Variables given for the experimental setup}
         \PY{n}{Avogadro\PYZus{}number} \PY{o}{=} \PY{l+m+mf}{6.022140857} \PY{o}{*} \PY{p}{(}\PY{l+m+mi}{10}\PY{o}{*}\PY{o}{*}\PY{l+m+mi}{23}\PY{p}{)}                 \PY{c+c1}{\PYZsh{}Avogadro\PYZsq{}s Number of Molecules per mole}
         \PY{n}{reaction\PYZus{}volume} \PY{o}{=} \PY{l+m+mi}{150}                                    \PY{c+c1}{\PYZsh{}150 uL reaction volume}
         \PY{n}{number\PYZus{}of\PYZus{}Snap\PYZus{}Tcells} \PY{o}{=} \PY{l+m+mi}{150000}                           \PY{c+c1}{\PYZsh{}150,000 T Cells in reaction}
         \PY{n}{number\PYZus{}of\PYZus{}tumor\PYZus{}cells} \PY{o}{=} \PY{l+m+mi}{400000}                           \PY{c+c1}{\PYZsh{}400,000 Tumor Cells in reaction      }
         
         \PY{c+c1}{\PYZsh{}Calculating the initial molecular concentrations}
         \PY{n}{Concentration\PYZus{}of\PYZus{}T\PYZus{}Cells} \PY{o}{=} \PY{n}{number\PYZus{}of\PYZus{}Snap\PYZus{}Tcells} \PY{o}{*} \PY{n}{number\PYZus{}of\PYZus{}receptors\PYZus{}per\PYZus{}Tcell} \PY{o}{*} \PYZbs{}
                                        \PY{p}{(}\PY{l+m+mi}{10}\PY{o}{*}\PY{o}{*}\PY{l+m+mi}{6}\PY{p}{)} \PY{o}{*} \PY{p}{(}\PY{l+m+mi}{10}\PY{o}{*}\PY{o}{*}\PY{l+m+mi}{9}\PY{p}{)} \PY{o}{*} \PY{p}{(}\PY{l+m+mi}{1}\PY{o}{/}\PY{n}{Avogadro\PYZus{}number}\PY{p}{)} \PY{o}{*} \PY{p}{(}\PY{l+m+mi}{1}\PY{o}{/}\PY{n}{reaction\PYZus{}volume}\PY{p}{)}
         \PY{n}{Concentration\PYZus{}of\PYZus{}Tumor\PYZus{}Cells} \PY{o}{=} \PY{n}{number\PYZus{}of\PYZus{}tumor\PYZus{}cells} \PY{o}{*} \PY{n}{number\PYZus{}of\PYZus{}receptors\PYZus{}per\PYZus{}tumor} \PY{o}{*} \PYZbs{}
                                        \PY{p}{(}\PY{l+m+mi}{10}\PY{o}{*}\PY{o}{*}\PY{l+m+mi}{6}\PY{p}{)} \PY{o}{*} \PY{p}{(}\PY{l+m+mi}{10}\PY{o}{*}\PY{o}{*}\PY{l+m+mi}{9}\PY{p}{)} \PY{o}{*} \PY{p}{(}\PY{l+m+mi}{1}\PY{o}{/}\PY{n}{Avogadro\PYZus{}number}\PY{p}{)} \PY{o}{*} \PY{p}{(}\PY{l+m+mi}{1}\PY{o}{/}\PY{n}{reaction\PYZus{}volume}\PY{p}{)}
         \PY{n}{Concentration\PYZus{}of\PYZus{}Antibody} \PY{o}{=} \PY{l+m+mi}{1} \PY{c+c1}{\PYZsh{}Necessary to initialize the simulation, actual value is a range}
         
         
         \PY{c+c1}{\PYZsh{}Saving the variables for simulation}
         \PY{n}{Initial} \PY{o}{=} \PY{p}{[}\PY{n}{Concentration\PYZus{}of\PYZus{}T\PYZus{}Cells}\PY{p}{,}\PY{n}{Concentration\PYZus{}of\PYZus{}Antibody}\PY{p}{,}\PY{n}{Concentration\PYZus{}of\PYZus{}Tumor\PYZus{}Cells}\PY{p}{,}\PY{l+m+mi}{0}\PY{p}{,}\PY{l+m+mi}{0}\PY{p}{,}\PY{l+m+mi}{0}\PY{p}{]}
         \PY{n}{parameter\PYZus{}range} \PY{o}{=} \PY{n}{np}\PY{o}{.}\PY{n}{logspace}\PY{p}{(}\PY{n}{log\PYZus{}lower\PYZus{}bound}\PY{p}{,}\PY{n}{log\PYZus{}upper\PYZus{}bound}\PY{p}{,}\PY{n}{num}\PY{o}{=}\PY{n}{step\PYZus{}size}\PY{p}{,} \PY{n}{endpoint}\PY{o}{=}\PY{k+kc}{True}\PY{p}{,} \PY{n}{base}\PY{o}{=}\PY{l+m+mf}{10.0}\PY{p}{)}
         \PY{n}{time} \PY{o}{=} \PY{n}{np}\PY{o}{.}\PY{n}{linspace}\PY{p}{(}\PY{l+m+mi}{0}\PY{p}{,}\PY{l+m+mi}{100000000}\PY{p}{,}\PY{l+m+mi}{10000}\PY{p}{)}
\end{Verbatim}


    \hypertarget{the-function-of-odes-for-simulation-copied-from-the-setup-file}{%
\subsubsection{The function of ODEs for simulation (copied from the
setup
file)}\label{the-function-of-odes-for-simulation-copied-from-the-setup-file}}

Here is where the rates, concentrations, and reactions are defined for
simulation. Due to the nature of this function, only initial
concentrations (C) and time (t) can be sent to the function. A global
variable (k) with all the calculated rate constants are sent separately.
The function returns the rates of change for each concentration.

    \begin{Verbatim}[commandchars=\\\{\}]
{\color{incolor}In [{\color{incolor}32}]:} \PY{k}{def} \PY{n+nf}{rxn}\PY{p}{(}\PY{n}{C}\PY{p}{,}\PY{n}{t}\PY{p}{)}\PY{p}{:}
             \PY{c+c1}{\PYZsh{}Loading the global variable reaction rates}
             \PY{n}{kf1} \PY{o}{=} \PY{n}{k}\PY{p}{[}\PY{l+m+mi}{0}\PY{p}{]}                                             
             \PY{n}{kr1} \PY{o}{=} \PY{n}{k}\PY{p}{[}\PY{l+m+mi}{1}\PY{p}{]}                                             
             \PY{n}{kf2} \PY{o}{=} \PY{n}{k}\PY{p}{[}\PY{l+m+mi}{2}\PY{p}{]}                                             
             \PY{n}{kr2} \PY{o}{=} \PY{n}{k}\PY{p}{[}\PY{l+m+mi}{3}\PY{p}{]}                                             
             \PY{n}{kf3} \PY{o}{=} \PY{n}{k}\PY{p}{[}\PY{l+m+mi}{4}\PY{p}{]}                                             
             \PY{n}{kr3} \PY{o}{=} \PY{n}{k}\PY{p}{[}\PY{l+m+mi}{5}\PY{p}{]}                                             
             \PY{n}{kf4} \PY{o}{=} \PY{n}{k}\PY{p}{[}\PY{l+m+mi}{6}\PY{p}{]}                                             
             \PY{n}{kr4} \PY{o}{=} \PY{n}{k}\PY{p}{[}\PY{l+m+mi}{7}\PY{p}{]}        
             
             \PY{c+c1}{\PYZsh{}Loading the initial concentrations}
             \PY{n}{C\PYZus{}snap} \PY{o}{=} \PY{n}{C}\PY{p}{[}\PY{l+m+mi}{0}\PY{p}{]}                   \PY{c+c1}{\PYZsh{}initial concentration of free T Cell receptor}
             \PY{n}{C\PYZus{}anti} \PY{o}{=} \PY{n}{C}\PY{p}{[}\PY{l+m+mi}{1}\PY{p}{]}                   \PY{c+c1}{\PYZsh{}initial concentration of free antibody}
             \PY{n}{C\PYZus{}tumo} \PY{o}{=} \PY{n}{C}\PY{p}{[}\PY{l+m+mi}{2}\PY{p}{]}                   \PY{c+c1}{\PYZsh{}initial concentration of free Tumor cell receptor}
             \PY{n}{C\PYZus{}snap\PYZus{}anti} \PY{o}{=} \PY{n}{C}\PY{p}{[}\PY{l+m+mi}{3}\PY{p}{]}              \PY{c+c1}{\PYZsh{}initial concentration of bound T Cell \PYZhy{} Antibody}
             \PY{n}{C\PYZus{}anti\PYZus{}tumo} \PY{o}{=} \PY{n}{C}\PY{p}{[}\PY{l+m+mi}{4}\PY{p}{]}              \PY{c+c1}{\PYZsh{}initial concentration of bound Tumor Cell \PYZhy{} Antibody}
             \PY{n}{C\PYZus{}snap\PYZus{}anti\PYZus{}tumo} \PY{o}{=} \PY{n}{C}\PY{p}{[}\PY{l+m+mi}{5}\PY{p}{]}         \PY{c+c1}{\PYZsh{}initial concentration of Ternary Body}
             
             \PY{c+c1}{\PYZsh{}Separation of key equation terms for simplicity}
             \PY{n}{term1} \PY{o}{=} \PY{n}{kf1} \PY{o}{*} \PY{n}{C\PYZus{}snap} \PY{o}{*} \PY{n}{C\PYZus{}anti}           \PY{c+c1}{\PYZsh{}binding of Tcell to antibody                              }
             \PY{n}{term2} \PY{o}{=} \PY{n}{kr1} \PY{o}{*} \PY{n}{C\PYZus{}snap\PYZus{}anti}               \PY{c+c1}{\PYZsh{}dissociation of Tcell \PYZhy{} antibody}
             \PY{n}{term3} \PY{o}{=} \PY{n}{kf2} \PY{o}{*} \PY{n}{C\PYZus{}anti} \PY{o}{*} \PY{n}{C\PYZus{}tumo}           \PY{c+c1}{\PYZsh{}binding of TumorCell to antibody}
             \PY{n}{term4} \PY{o}{=} \PY{n}{kr2} \PY{o}{*} \PY{n}{C\PYZus{}anti\PYZus{}tumo}               \PY{c+c1}{\PYZsh{}dissociation of TumorCell \PYZhy{} antibody}
             \PY{n}{term5} \PY{o}{=} \PY{n}{kf3} \PY{o}{*} \PY{n}{C\PYZus{}snap\PYZus{}anti} \PY{o}{*} \PY{n}{C\PYZus{}tumo}      \PY{c+c1}{\PYZsh{}binding of Tcell\PYZhy{}antibody to Tumor Cell}
             \PY{n}{term6} \PY{o}{=} \PY{n}{kr3} \PY{o}{*} \PY{n}{C\PYZus{}snap\PYZus{}anti\PYZus{}tumo}          \PY{c+c1}{\PYZsh{}dissociation of Ternary Body}
             \PY{n}{term7} \PY{o}{=} \PY{n}{kf4} \PY{o}{*} \PY{n}{C\PYZus{}anti\PYZus{}tumo} \PY{o}{*} \PY{n}{C\PYZus{}snap}      \PY{c+c1}{\PYZsh{}binding of TumorCell\PYZhy{}antibody to T Cell}
             \PY{n}{term8} \PY{o}{=} \PY{n}{kr4} \PY{o}{*} \PY{n}{C\PYZus{}snap\PYZus{}anti\PYZus{}tumo}          \PY{c+c1}{\PYZsh{}dissociation of Ternary Body}
             
             \PY{c+c1}{\PYZsh{}ODEs that model the dynamic change in concentration of each chemical species}
             \PY{n}{d\PYZus{}C\PYZus{}snap\PYZus{}dt} \PY{o}{=} \PY{o}{\PYZhy{}} \PY{n}{term1} \PY{o}{+} \PY{n}{term2} \PY{o}{\PYZhy{}} \PY{n}{term7} \PY{o}{+} \PY{n}{term8}           \PY{c+c1}{\PYZsh{}change in free T Cell receptor}
             \PY{n}{d\PYZus{}C\PYZus{}anti\PYZus{}dt} \PY{o}{=} \PY{o}{\PYZhy{}} \PY{n}{term1} \PY{o}{+} \PY{n}{term2} \PY{o}{\PYZhy{}} \PY{n}{term3} \PY{o}{+} \PY{n}{term4}           \PY{c+c1}{\PYZsh{}change in free antibody}
             \PY{n}{d\PYZus{}C\PYZus{}tumo\PYZus{}dt} \PY{o}{=} \PY{o}{\PYZhy{}} \PY{n}{term3} \PY{o}{+} \PY{n}{term4} \PY{o}{\PYZhy{}} \PY{n}{term5} \PY{o}{+} \PY{n}{term6}           \PY{c+c1}{\PYZsh{}change in free Tumor cell receptor}
             \PY{n}{d\PYZus{}C\PYZus{}snap\PYZus{}anti\PYZus{}dt} \PY{o}{=} \PY{o}{+} \PY{n}{term1} \PY{o}{\PYZhy{}} \PY{n}{term2} \PY{o}{\PYZhy{}} \PY{n}{term5} \PY{o}{+} \PY{n}{term6}      \PY{c+c1}{\PYZsh{}change in bound T Cell \PYZhy{} Antibody}
             \PY{n}{d\PYZus{}C\PYZus{}anti\PYZus{}tumo\PYZus{}dt} \PY{o}{=} \PY{o}{+} \PY{n}{term3} \PY{o}{\PYZhy{}} \PY{n}{term4} \PY{o}{\PYZhy{}} \PY{n}{term7} \PY{o}{+} \PY{n}{term8}      \PY{c+c1}{\PYZsh{}change in bound Tumor Cell \PYZhy{} Antibody}
             \PY{n}{d\PYZus{}C\PYZus{}snap\PYZus{}anti\PYZus{}tumo\PYZus{}dt} \PY{o}{=} \PY{o}{+} \PY{n}{term5} \PY{o}{\PYZhy{}} \PY{n}{term6} \PY{o}{+} \PY{n}{term7} \PY{o}{\PYZhy{}} \PY{n}{term8} \PY{c+c1}{\PYZsh{}change in Ternary Body}
             
             \PY{k}{return}\PY{p}{(}\PY{p}{[}\PY{n}{d\PYZus{}C\PYZus{}snap\PYZus{}dt}\PY{p}{,}\PY{n}{d\PYZus{}C\PYZus{}anti\PYZus{}dt}\PY{p}{,}\PY{n}{d\PYZus{}C\PYZus{}tumo\PYZus{}dt}\PY{p}{,}\PY{n}{d\PYZus{}C\PYZus{}snap\PYZus{}anti\PYZus{}dt}\PY{p}{,}\PY{n}{d\PYZus{}C\PYZus{}anti\PYZus{}tumo\PYZus{}dt}\PY{p}{,}\PY{n}{d\PYZus{}C\PYZus{}snap\PYZus{}anti\PYZus{}tumo\PYZus{}dt}\PY{p}{]}\PY{p}{)}
\end{Verbatim}


    \hypertarget{simulation-and-plotting-equilibrium-concentrations-over-a-range-of-antibody-release-rates}{%
\subsubsection{Simulation and Plotting Equilibrium Concentrations over a
Range of Antibody Release
Rates}\label{simulation-and-plotting-equilibrium-concentrations-over-a-range-of-antibody-release-rates}}

In this analysis, we simulate the same situation as above but varying
the antibody unbinding rate. We will be scanning a logarithmic space
from 10e-12 to 10e-6. Please note that the analysis above took place at
Kd of 15 * 10e-9 (or 15 nM).

    \begin{Verbatim}[commandchars=\\\{\}]
{\color{incolor}In [{\color{incolor}85}]:} \PY{c+c1}{\PYZsh{}creating an empty list to store simulation data}
         \PY{n}{data} \PY{o}{=} \PY{p}{[}\PY{p}{]}
         
         \PY{c+c1}{\PYZsh{}the for look to go over the range of the parameter scan}
         \PY{k}{for} \PY{n}{dissociation\PYZus{}rate} \PY{o+ow}{in} \PY{n}{parameter\PYZus{}range}\PY{p}{:}
             \PY{n}{Antibody\PYZus{}dissociation} \PY{o}{=} \PY{n}{dissociation\PYZus{}rate}
             \PY{c+c1}{\PYZsh{}Calculating the Kinetic Parameters \PYZhy{} since we only have equilibrium data, we are interested in ratios, not numbers}
             \PY{n}{k\PYZus{}binding\PYZus{}snap\PYZus{}to\PYZus{}antibody} \PY{o}{=} \PY{l+m+mi}{10}\PY{o}{*}\PY{o}{*}\PY{o}{\PYZhy{}}\PY{l+m+mi}{5}                      \PY{c+c1}{\PYZsh{}}
             \PY{n}{k\PYZus{}release\PYZus{}snap\PYZus{}to\PYZus{}antibody} \PY{o}{=} \PY{n}{Antibody\PYZus{}dissociation}  \PY{c+c1}{\PYZsh{}Scanning the dissociation rate}
             \PY{n}{k\PYZus{}binding\PYZus{}antibody\PYZus{}to\PYZus{}tumor} \PY{o}{=} \PY{l+m+mi}{10}\PY{o}{*}\PY{o}{*}\PY{o}{\PYZhy{}}\PY{l+m+mi}{5} 
             \PY{n}{k\PYZus{}release\PYZus{}antibody\PYZus{}to\PYZus{}tumor} \PY{o}{=} \PY{n}{k\PYZus{}binding\PYZus{}antibody\PYZus{}to\PYZus{}tumor} \PY{o}{*} \PY{n}{Antibody\PYZus{}dissociation} \PY{o}{*} \PY{l+m+mi}{10}\PY{o}{*}\PY{o}{*}\PY{o}{\PYZhy{}}\PY{l+m+mi}{9}
             \PY{n}{k\PYZus{}cooperativity\PYZus{}binding\PYZus{}snap} \PY{o}{=} \PY{n}{k\PYZus{}binding\PYZus{}snap\PYZus{}to\PYZus{}antibody} \PY{o}{*} \PY{n}{alpha}
             \PY{n}{k\PYZus{}cooperativity\PYZus{}releasing\PYZus{}snap} \PY{o}{=} \PY{n}{k\PYZus{}release\PYZus{}snap\PYZus{}to\PYZus{}antibody} \PY{o}{*} \PY{n}{alpha}
             \PY{n}{k\PYZus{}cooperativity\PYZus{}binding\PYZus{}tumor} \PY{o}{=} \PY{n}{k\PYZus{}binding\PYZus{}antibody\PYZus{}to\PYZus{}tumor} \PY{o}{*} \PY{n}{alpha}
             \PY{n}{k\PYZus{}cooperativity\PYZus{}releasing\PYZus{}tumor} \PY{o}{=} \PY{n}{k\PYZus{}release\PYZus{}antibody\PYZus{}to\PYZus{}tumor} \PY{o}{*} \PY{n}{alpha}
         
             \PY{n}{k} \PY{o}{=} \PY{p}{[}\PY{n}{k\PYZus{}binding\PYZus{}snap\PYZus{}to\PYZus{}antibody}\PY{p}{,}\PY{n}{k\PYZus{}release\PYZus{}snap\PYZus{}to\PYZus{}antibody}\PY{p}{,}\PY{n}{k\PYZus{}binding\PYZus{}antibody\PYZus{}to\PYZus{}tumor}\PY{p}{,}\PYZbs{}
                  \PY{n}{k\PYZus{}release\PYZus{}antibody\PYZus{}to\PYZus{}tumor}\PY{p}{,}\PY{n}{k\PYZus{}cooperativity\PYZus{}binding\PYZus{}snap}\PY{p}{,}\PY{n}{k\PYZus{}cooperativity\PYZus{}releasing\PYZus{}snap}\PY{p}{,}\PYZbs{}
                  \PY{n}{k\PYZus{}cooperativity\PYZus{}binding\PYZus{}tumor}\PY{p}{,}\PY{n}{k\PYZus{}cooperativity\PYZus{}releasing\PYZus{}tumor}\PY{p}{]}
             
             \PY{n}{N} \PY{o}{=} \PY{l+m+mi}{100}
             \PY{n}{antibody\PYZus{}range} \PY{o}{=} \PY{n}{np}\PY{o}{.}\PY{n}{logspace}\PY{p}{(}\PY{o}{\PYZhy{}}\PY{l+m+mi}{3}\PY{p}{,}\PY{l+m+mi}{1}\PY{p}{,}\PY{n}{N}\PY{p}{)}
             \PY{n}{ternary\PYZus{}body} \PY{o}{=} \PY{p}{[}\PY{p}{]}
             
             \PY{c+c1}{\PYZsh{}the for loop to go over the range of the antibody concentrations}
             \PY{k}{for} \PY{n}{i} \PY{o+ow}{in} \PY{n}{antibody\PYZus{}range}\PY{p}{:}
                 \PY{n}{Initial}\PY{p}{[}\PY{l+m+mi}{1}\PY{p}{]} \PY{o}{=} \PY{n}{i}
                 \PY{n}{C} \PY{o}{=} \PY{n}{odeint}\PY{p}{(}\PY{n}{rxn}\PY{p}{,}\PY{n}{Initial}\PY{p}{,}\PY{n}{time}\PY{p}{)}
                 \PY{n}{ternary\PYZus{}body}\PY{o}{.}\PY{n}{append}\PY{p}{(}\PY{n+nb}{float}\PY{p}{(}\PY{p}{(}\PY{n}{C}\PY{p}{[}\PY{l+m+mi}{100}\PY{p}{,}\PY{l+m+mi}{5}\PY{p}{]}\PY{p}{)}\PY{p}{)}\PY{p}{)}
             \PY{n}{data}\PY{o}{.}\PY{n}{append}\PY{p}{(}\PY{n}{ternary\PYZus{}body}\PY{p}{)}
\end{Verbatim}


    \begin{Verbatim}[commandchars=\\\{\}]
{\color{incolor}In [{\color{incolor}86}]:} \PY{c+c1}{\PYZsh{}Create the figure and set the layout}
         \PY{n}{fig} \PY{o}{=} \PY{n}{plt}\PY{o}{.}\PY{n}{figure}\PY{p}{(}\PY{n}{figsize}\PY{o}{=}\PY{p}{(}\PY{l+m+mi}{8}\PY{p}{,} \PY{l+m+mi}{4}\PY{p}{)}\PY{p}{,} \PY{n}{dpi}\PY{o}{=}\PY{l+m+mi}{100}\PY{p}{)}
         \PY{n}{gs1} \PY{o}{=} \PY{n}{gridspec}\PY{o}{.}\PY{n}{GridSpec}\PY{p}{(}\PY{l+m+mi}{1}\PY{p}{,} \PY{l+m+mi}{1}\PY{p}{)}
         \PY{n}{ax} \PY{o}{=} \PY{n}{fig}\PY{o}{.}\PY{n}{add\PYZus{}subplot}\PY{p}{(}\PY{n}{gs1}\PY{p}{[}\PY{l+m+mi}{0}\PY{p}{]}\PY{p}{)}
         \PY{n}{plt}\PY{o}{.}\PY{n}{xlabel}\PY{p}{(}\PY{l+s+s1}{\PYZsq{}}\PY{l+s+s1}{Concentration of Antibody (nM)}\PY{l+s+s1}{\PYZsq{}}\PY{p}{)}
         \PY{n}{plt}\PY{o}{.}\PY{n}{ylabel}\PY{p}{(}\PY{l+s+s1}{\PYZsq{}}\PY{l+s+s1}{Concentration of ABC (nM)}\PY{l+s+s1}{\PYZsq{}}\PY{p}{)}
         \PY{n}{plt}\PY{o}{.}\PY{n}{xscale}\PY{p}{(}\PY{l+s+s2}{\PYZdq{}}\PY{l+s+s2}{log}\PY{l+s+s2}{\PYZdq{}}\PY{p}{)}
         
         \PY{c+c1}{\PYZsh{}Create a legend for the figure}
         \PY{n}{legend\PYZus{}list} \PY{o}{=} \PY{p}{[}\PY{p}{]}
         \PY{k}{for} \PY{n}{i} \PY{o+ow}{in} \PY{n+nb}{range}\PY{p}{(}\PY{l+m+mi}{0}\PY{p}{,}\PY{n}{step\PYZus{}size}\PY{p}{)}\PY{p}{:}
             \PY{n}{ax}\PY{o}{.}\PY{n}{plot}\PY{p}{(}\PY{n}{antibody\PYZus{}range}\PY{p}{,}\PY{n}{data}\PY{p}{[}\PY{n}{i}\PY{p}{]}\PY{p}{,}\PY{n}{linewidth}\PY{o}{=}\PY{l+m+mf}{2.0}\PY{p}{)}
             \PY{n}{legend\PYZus{}list}\PY{o}{.}\PY{n}{append}\PY{p}{(}\PY{l+s+s1}{\PYZsq{}}\PY{l+s+s1}{Kd = }\PY{l+s+s1}{\PYZsq{}}\PY{o}{+} \PY{l+s+s1}{\PYZsq{}}\PY{l+s+si}{\PYZob{}:.0e\PYZcb{}}\PY{l+s+s1}{\PYZsq{}}\PY{o}{.}\PY{n}{format}\PY{p}{(}\PY{n}{parameter\PYZus{}range}\PY{p}{[}\PY{n}{i}\PY{p}{]}\PY{p}{)} \PY{o}{+} \PY{l+s+s1}{\PYZsq{}}\PY{l+s+s1}{ M}\PY{l+s+s1}{\PYZsq{}}\PY{p}{)}
         
         \PY{c+c1}{\PYZsh{} Shrink current axis by 20\PYZpc{}}
         \PY{n}{box} \PY{o}{=} \PY{n}{ax}\PY{o}{.}\PY{n}{get\PYZus{}position}\PY{p}{(}\PY{p}{)}
         \PY{n}{ax}\PY{o}{.}\PY{n}{set\PYZus{}position}\PY{p}{(}\PY{p}{[}\PY{n}{box}\PY{o}{.}\PY{n}{x0}\PY{p}{,} \PY{n}{box}\PY{o}{.}\PY{n}{y0}\PY{p}{,} \PY{n}{box}\PY{o}{.}\PY{n}{width} \PY{o}{*} \PY{l+m+mf}{0.8}\PY{p}{,} \PY{n}{box}\PY{o}{.}\PY{n}{height}\PY{p}{]}\PY{p}{)}
         
         \PY{c+c1}{\PYZsh{} Put a legend to the right of the current axis}
         \PY{n}{ax}\PY{o}{.}\PY{n}{legend}\PY{p}{(}\PY{n}{legend\PYZus{}list}\PY{p}{,}\PY{n}{loc}\PY{o}{=}\PY{l+s+s1}{\PYZsq{}}\PY{l+s+s1}{center left}\PY{l+s+s1}{\PYZsq{}}\PY{p}{,} \PY{n}{bbox\PYZus{}to\PYZus{}anchor}\PY{o}{=}\PY{p}{(}\PY{l+m+mi}{1}\PY{p}{,} \PY{l+m+mf}{0.5}\PY{p}{)}\PY{p}{)}
         
         \PY{c+c1}{\PYZsh{}Creating the plot, saving it, and showing it}
         \PY{n}{gs1}\PY{o}{.}\PY{n}{tight\PYZus{}layout}\PY{p}{(}\PY{n}{fig}\PY{p}{)}
         \PY{n}{gs1}\PY{o}{.}\PY{n}{tight\PYZus{}layout}\PY{p}{(}\PY{n}{fig}\PY{p}{,} \PY{n}{rect}\PY{o}{=}\PY{p}{[}\PY{l+m+mi}{0}\PY{p}{,} \PY{l+m+mi}{0}\PY{p}{,} \PY{l+m+mi}{1}\PY{p}{,} \PY{l+m+mi}{1}\PY{p}{]}\PY{p}{)}
         \PY{n}{plt}\PY{o}{.}\PY{n}{savefig}\PY{p}{(}\PY{l+s+s1}{\PYZsq{}}\PY{l+s+s1}{Figures/Figure3\PYZus{}ParameterScan\PYZus{}AntibodyDissociation.png}\PY{l+s+s1}{\PYZsq{}}\PY{p}{)}
         \PY{n}{plt}\PY{o}{.}\PY{n}{show}\PY{p}{(}\PY{p}{)}
\end{Verbatim}


    \begin{center}
    \adjustimage{max size={0.9\linewidth}{0.9\paperheight}}{output_10_0.png}
    \end{center}
    { \hspace*{\fill} \\}
    
    \hypertarget{plotting-this-relationship-in-3d}{%
\subsubsection{Plotting this relationship in
3D}\label{plotting-this-relationship-in-3d}}

Here we re-examine the relationship between antibody concentration and
Kd in a 3D plot.

    \begin{Verbatim}[commandchars=\\\{\}]
{\color{incolor}In [{\color{incolor}28}]:} \PY{k+kn}{from} \PY{n+nn}{matplotlib} \PY{k}{import} \PY{n}{cm}
         \PY{k+kn}{from} \PY{n+nn}{mpl\PYZus{}toolkits}\PY{n+nn}{.}\PY{n+nn}{mplot3d} \PY{k}{import} \PY{n}{Axes3D}
         \PY{k+kn}{from} \PY{n+nn}{matplotlib}\PY{n+nn}{.}\PY{n+nn}{ticker} \PY{k}{import} \PY{n}{LinearLocator}\PY{p}{,} \PY{n}{FormatStrFormatter}
\end{Verbatim}


    \begin{Verbatim}[commandchars=\\\{\}]
{\color{incolor}In [{\color{incolor}153}]:} \PY{c+c1}{\PYZsh{} Simulate the model over a range}
          \PY{n}{N} \PY{o}{=} \PY{l+m+mi}{30}
          \PY{n}{antibody\PYZus{}range} \PY{o}{=} \PY{n}{np}\PY{o}{.}\PY{n}{logspace}\PY{p}{(}\PY{o}{\PYZhy{}}\PY{l+m+mi}{3}\PY{p}{,}\PY{l+m+mi}{1}\PY{p}{,}\PY{n}{N}\PY{p}{)}
          \PY{n}{Kd\PYZus{}range} \PY{o}{=} \PY{n}{np}\PY{o}{.}\PY{n}{logspace}\PY{p}{(}\PY{o}{\PYZhy{}}\PY{l+m+mi}{9}\PY{p}{,}\PY{o}{\PYZhy{}}\PY{l+m+mi}{3}\PY{p}{,}\PY{n}{N}\PY{p}{)}
          \PY{n}{time} \PY{o}{=} \PY{n}{np}\PY{o}{.}\PY{n}{linspace}\PY{p}{(}\PY{l+m+mi}{0}\PY{p}{,}\PY{l+m+mi}{100000000}\PY{p}{,}\PY{l+m+mi}{10000}\PY{p}{)}
          \PY{n}{X}\PY{p}{,} \PY{n}{Y} \PY{o}{=} \PY{n}{np}\PY{o}{.}\PY{n}{meshgrid}\PY{p}{(}\PY{n}{antibody\PYZus{}range}\PY{p}{,} \PY{n}{Kd\PYZus{}range}\PY{p}{)}
          \PY{n}{Z} \PY{o}{=} \PY{n}{np}\PY{o}{.}\PY{n}{zeros}\PY{p}{(}\PY{n}{X}\PY{o}{.}\PY{n}{shape}\PY{p}{)}
          
          \PY{n}{row} \PY{o}{=} \PY{l+m+mi}{0}
          \PY{n}{col} \PY{o}{=} \PY{l+m+mi}{0}
          
          \PY{k}{for} \PY{n}{Kd} \PY{o+ow}{in} \PY{n}{Kd\PYZus{}range}\PY{p}{:}
              \PY{n}{Antibody\PYZus{}dissociation} \PY{o}{=} \PY{n}{Kd}
              \PY{c+c1}{\PYZsh{}Calculating the Kinetic Parameters \PYZhy{} since we only have equilibrium data, we are interested in ratios, not numbers}
              \PY{n}{k\PYZus{}binding\PYZus{}snap\PYZus{}to\PYZus{}antibody} \PY{o}{=} \PY{l+m+mi}{10}\PY{o}{*}\PY{o}{*}\PY{o}{\PYZhy{}}\PY{l+m+mi}{5}                      \PY{c+c1}{\PYZsh{}}
              \PY{n}{k\PYZus{}release\PYZus{}snap\PYZus{}to\PYZus{}antibody} \PY{o}{=} \PY{n}{Antibody\PYZus{}dissociation}  \PY{c+c1}{\PYZsh{}Scanning the dissociation rate}
              \PY{n}{k\PYZus{}binding\PYZus{}antibody\PYZus{}to\PYZus{}tumor} \PY{o}{=} \PY{l+m+mi}{10}\PY{o}{*}\PY{o}{*}\PY{o}{\PYZhy{}}\PY{l+m+mi}{5} 
              \PY{n}{k\PYZus{}release\PYZus{}antibody\PYZus{}to\PYZus{}tumor} \PY{o}{=} \PY{n}{k\PYZus{}binding\PYZus{}antibody\PYZus{}to\PYZus{}tumor} \PY{o}{*} \PY{n}{Antibody\PYZus{}dissociation} \PY{o}{*} \PY{l+m+mi}{10}\PY{o}{*}\PY{o}{*}\PY{o}{\PYZhy{}}\PY{l+m+mi}{9}
              \PY{n}{k\PYZus{}cooperativity\PYZus{}binding\PYZus{}snap} \PY{o}{=} \PY{n}{k\PYZus{}binding\PYZus{}snap\PYZus{}to\PYZus{}antibody} \PY{o}{*} \PY{n}{alpha}
              \PY{n}{k\PYZus{}cooperativity\PYZus{}releasing\PYZus{}snap} \PY{o}{=} \PY{n}{k\PYZus{}release\PYZus{}snap\PYZus{}to\PYZus{}antibody} \PY{o}{*} \PY{n}{alpha}
              \PY{n}{k\PYZus{}cooperativity\PYZus{}binding\PYZus{}tumor} \PY{o}{=} \PY{n}{k\PYZus{}binding\PYZus{}antibody\PYZus{}to\PYZus{}tumor} \PY{o}{*} \PY{n}{alpha}
              \PY{n}{k\PYZus{}cooperativity\PYZus{}releasing\PYZus{}tumor} \PY{o}{=} \PY{n}{k\PYZus{}release\PYZus{}antibody\PYZus{}to\PYZus{}tumor} \PY{o}{*} \PY{n}{alpha}
          
              \PY{n}{k} \PY{o}{=} \PY{p}{[}\PY{n}{k\PYZus{}binding\PYZus{}snap\PYZus{}to\PYZus{}antibody}\PY{p}{,}\PY{n}{k\PYZus{}release\PYZus{}snap\PYZus{}to\PYZus{}antibody}\PY{p}{,}\PY{n}{k\PYZus{}binding\PYZus{}antibody\PYZus{}to\PYZus{}tumor}\PY{p}{,}\PYZbs{}
                   \PY{n}{k\PYZus{}release\PYZus{}antibody\PYZus{}to\PYZus{}tumor}\PY{p}{,}\PY{n}{k\PYZus{}cooperativity\PYZus{}binding\PYZus{}snap}\PY{p}{,}\PY{n}{k\PYZus{}cooperativity\PYZus{}releasing\PYZus{}snap}\PY{p}{,}\PYZbs{}
                   \PY{n}{k\PYZus{}cooperativity\PYZus{}binding\PYZus{}tumor}\PY{p}{,}\PY{n}{k\PYZus{}cooperativity\PYZus{}releasing\PYZus{}tumor}\PY{p}{]}
              
              \PY{c+c1}{\PYZsh{}the for loop to go over the range of the antibody concentrations}
              \PY{k}{for} \PY{n}{antibody} \PY{o+ow}{in} \PY{n}{antibody\PYZus{}range}\PY{p}{:}
                  \PY{n}{Initial}\PY{p}{[}\PY{l+m+mi}{1}\PY{p}{]} \PY{o}{=} \PY{n}{antibody}
                  \PY{n}{C} \PY{o}{=} \PY{n}{odeint}\PY{p}{(}\PY{n}{rxn}\PY{p}{,}\PY{n}{Initial}\PY{p}{,}\PY{n}{time}\PY{p}{)}
                  \PY{n}{Z}\PY{p}{[}\PY{n}{row}\PY{p}{,}\PY{n}{col}\PY{p}{]} \PY{o}{=} \PY{n+nb}{float}\PY{p}{(}\PY{p}{(}\PY{n}{C}\PY{p}{[}\PY{l+m+mi}{100}\PY{p}{,}\PY{l+m+mi}{5}\PY{p}{]}\PY{p}{)}\PY{p}{)}        
                  \PY{n}{col} \PY{o}{+}\PY{o}{=} \PY{l+m+mi}{1}
              \PY{n}{col} \PY{o}{=} \PY{l+m+mi}{0}
              \PY{n}{row} \PY{o}{+}\PY{o}{=} \PY{l+m+mi}{1}
              
          \PY{c+c1}{\PYZsh{}Collapse our Logspace to Linspace because matplotlib has difficulty with logspace plotting}
          \PY{n}{lin\PYZus{}antibody} \PY{o}{=} \PY{n}{np}\PY{o}{.}\PY{n}{linspace}\PY{p}{(}\PY{o}{\PYZhy{}}\PY{l+m+mi}{3}\PY{p}{,}\PY{l+m+mi}{1}\PY{p}{,}\PY{n}{N}\PY{p}{)}
          \PY{n}{lin\PYZus{}Kd} \PY{o}{=} \PY{n}{np}\PY{o}{.}\PY{n}{linspace}\PY{p}{(}\PY{o}{\PYZhy{}}\PY{l+m+mi}{9}\PY{p}{,}\PY{o}{\PYZhy{}}\PY{l+m+mi}{3}\PY{p}{,}\PY{n}{N}\PY{p}{)}
          \PY{n}{lin\PYZus{}X}\PY{p}{,} \PY{n}{lin\PYZus{}Y} \PY{o}{=} \PY{n}{np}\PY{o}{.}\PY{n}{meshgrid}\PY{p}{(}\PY{n}{lin\PYZus{}antibody}\PY{p}{,} \PY{n}{lin\PYZus{}Kd}\PY{p}{)}
          
          \PY{c+c1}{\PYZsh{}Create the figure and set the layout}
          \PY{n}{fig} \PY{o}{=} \PY{n}{plt}\PY{o}{.}\PY{n}{figure}\PY{p}{(}\PY{n}{figsize}\PY{o}{=}\PY{p}{(}\PY{l+m+mi}{10}\PY{p}{,} \PY{l+m+mi}{10}\PY{p}{)}\PY{p}{,} \PY{n}{dpi}\PY{o}{=}\PY{l+m+mi}{100}\PY{p}{)}
          \PY{n}{ax} \PY{o}{=} \PY{n}{fig}\PY{o}{.}\PY{n}{gca}\PY{p}{(}\PY{n}{projection}\PY{o}{=}\PY{l+s+s1}{\PYZsq{}}\PY{l+s+s1}{3d}\PY{l+s+s1}{\PYZsq{}}\PY{p}{)}
          \PY{n}{plt}\PY{o}{.}\PY{n}{xlabel}\PY{p}{(}\PY{l+s+s1}{\PYZsq{}}\PY{l+s+s1}{Log [Antibody] (nM)}\PY{l+s+s1}{\PYZsq{}}\PY{p}{)}
          \PY{n}{plt}\PY{o}{.}\PY{n}{ylabel}\PY{p}{(}\PY{l+s+s1}{\PYZsq{}}\PY{l+s+s1}{Log Kd (M)}\PY{l+s+s1}{\PYZsq{}}\PY{p}{)}
          \PY{n}{ax}\PY{o}{.}\PY{n}{set\PYZus{}zlabel}\PY{p}{(}\PY{l+s+s1}{\PYZsq{}}\PY{l+s+s1}{Concentration of ABC (nM)}\PY{l+s+s1}{\PYZsq{}}\PY{p}{)}
          
          \PY{c+c1}{\PYZsh{} Plot the surface.}
          \PY{n}{surf} \PY{o}{=} \PY{n}{ax}\PY{o}{.}\PY{n}{plot\PYZus{}surface}\PY{p}{(}\PY{n}{lin\PYZus{}X}\PY{p}{,} \PY{n}{lin\PYZus{}Y}\PY{p}{,} \PY{n}{Z}\PY{p}{,} \PY{n}{cmap}\PY{o}{=}\PY{n}{cm}\PY{o}{.}\PY{n}{coolwarm}\PY{p}{,}
                                 \PY{n}{linewidth}\PY{o}{=}\PY{l+m+mi}{0}\PY{p}{,} \PY{n}{antialiased}\PY{o}{=}\PY{k+kc}{False}\PY{p}{)}
          
          \PY{c+c1}{\PYZsh{} Customize and details for the the axises}
          \PY{c+c1}{\PYZsh{} ax.set\PYZus{}xlim(\PYZhy{}5.01, 5.01)}
          \PY{c+c1}{\PYZsh{} ax.set\PYZus{}ylim(\PYZhy{}5.01, 5.01)}
          \PY{c+c1}{\PYZsh{} ax.set\PYZus{}zlim(\PYZhy{}1.01, 1.01)}
          
          \PY{n}{ax}\PY{o}{.}\PY{n}{zaxis}\PY{o}{.}\PY{n}{set\PYZus{}major\PYZus{}locator}\PY{p}{(}\PY{n}{LinearLocator}\PY{p}{(}\PY{l+m+mi}{10}\PY{p}{)}\PY{p}{)}
          \PY{n}{ax}\PY{o}{.}\PY{n}{zaxis}\PY{o}{.}\PY{n}{set\PYZus{}major\PYZus{}formatter}\PY{p}{(}\PY{n}{FormatStrFormatter}\PY{p}{(}\PY{l+s+s1}{\PYZsq{}}\PY{l+s+si}{\PYZpc{}.02f}\PY{l+s+s1}{\PYZsq{}}\PY{p}{)}\PY{p}{)}
          
          \PY{c+c1}{\PYZsh{} Add a color bar which maps values to colors}
          \PY{n}{fig}\PY{o}{.}\PY{n}{colorbar}\PY{p}{(}\PY{n}{surf}\PY{p}{,} \PY{n}{shrink}\PY{o}{=}\PY{l+m+mf}{0.5}\PY{p}{,} \PY{n}{aspect}\PY{o}{=}\PY{l+m+mi}{10}\PY{p}{)}
          
          \PY{c+c1}{\PYZsh{}Creating the plot, saving it, and showing it}
          \PY{c+c1}{\PYZsh{} ax.set\PYZus{}title(\PYZsq{}Test Surface\PYZsq{});}
          \PY{n}{ax}\PY{o}{.}\PY{n}{view\PYZus{}init}\PY{p}{(}\PY{l+m+mi}{10}\PY{p}{,} \PY{l+m+mi}{70}\PY{p}{)}
          \PY{n}{ax}\PY{o}{.}\PY{n}{invert\PYZus{}xaxis}\PY{p}{(}\PY{p}{)}
          \PY{n}{plt}\PY{o}{.}\PY{n}{savefig}\PY{p}{(}\PY{l+s+s1}{\PYZsq{}}\PY{l+s+s1}{Figures/Figure3\PYZus{}ParameterScan\PYZus{}AntibodyDissociation\PYZus{}3Dplot.png}\PY{l+s+s1}{\PYZsq{}}\PY{p}{)}
          \PY{n}{plt}\PY{o}{.}\PY{n}{show}\PY{p}{(}\PY{p}{)}
\end{Verbatim}


    \begin{center}
    \adjustimage{max size={0.9\linewidth}{0.9\paperheight}}{output_13_0.png}
    \end{center}
    { \hspace*{\fill} \\}
    
    \hypertarget{slowly-rotating-the-figure}{%
\subsubsection{Slowly Rotating the
Figure}\label{slowly-rotating-the-figure}}

Unfortunately, python/jupyter notbooks make it a little difficult to
visualize and interface with 3D plots. Over the next few graphs, we will
just be rotating the 3D plot to allow for a better view of the parameter
scan. We would like to note that by just looking at the increase in
antibody concentration leads to the characteristic bell-shaped curve
indicative of ternary body formation while the decrease of Kd (the
dissociation rate at equilibrium between the antibody and the T Cell)
leads to much higher ternary body concentrations (pointing to higher
therapeutic efficacy).

    \begin{Verbatim}[commandchars=\\\{\}]
{\color{incolor}In [{\color{incolor}154}]:} \PY{n}{ax}\PY{o}{.}\PY{n}{view\PYZus{}init}\PY{p}{(}\PY{l+m+mi}{10}\PY{p}{,} \PY{l+m+mi}{92}\PY{p}{)}
          \PY{n}{fig}
\end{Verbatim}

\texttt{\color{outcolor}Out[{\color{outcolor}154}]:}
    
    \begin{center}
    \adjustimage{max size={0.9\linewidth}{0.9\paperheight}}{output_15_0.png}
    \end{center}
    { \hspace*{\fill} \\}
    

    \begin{Verbatim}[commandchars=\\\{\}]
{\color{incolor}In [{\color{incolor}155}]:} \PY{n}{ax}\PY{o}{.}\PY{n}{view\PYZus{}init}\PY{p}{(}\PY{l+m+mi}{10}\PY{p}{,} \PY{l+m+mi}{110}\PY{p}{)}
          \PY{n}{fig}
\end{Verbatim}

\texttt{\color{outcolor}Out[{\color{outcolor}155}]:}
    
    \begin{center}
    \adjustimage{max size={0.9\linewidth}{0.9\paperheight}}{output_16_0.png}
    \end{center}
    { \hspace*{\fill} \\}
    

    \begin{Verbatim}[commandchars=\\\{\}]
{\color{incolor}In [{\color{incolor}156}]:} \PY{n}{ax}\PY{o}{.}\PY{n}{view\PYZus{}init}\PY{p}{(}\PY{l+m+mi}{10}\PY{p}{,} \PY{l+m+mi}{150}\PY{p}{)}
          \PY{n}{fig}
\end{Verbatim}

\texttt{\color{outcolor}Out[{\color{outcolor}156}]:}
    
    \begin{center}
    \adjustimage{max size={0.9\linewidth}{0.9\paperheight}}{output_17_0.png}
    \end{center}
    { \hspace*{\fill} \\}
    


    % Add a bibliography block to the postdoc
    
    
    
    \end{document}
